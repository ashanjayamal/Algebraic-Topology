% Options for packages loaded elsewhere
\PassOptionsToPackage{unicode}{hyperref}
\PassOptionsToPackage{hyphens}{url}
\documentclass[
]{book}
\usepackage{xcolor}
\usepackage{amsmath,amssymb}
\setcounter{secnumdepth}{5}
\usepackage{iftex}
\ifPDFTeX
  \usepackage[T1]{fontenc}
  \usepackage[utf8]{inputenc}
  \usepackage{textcomp} % provide euro and other symbols
\else % if luatex or xetex
  \usepackage{unicode-math} % this also loads fontspec
  \defaultfontfeatures{Scale=MatchLowercase}
  \defaultfontfeatures[\rmfamily]{Ligatures=TeX,Scale=1}
\fi
\usepackage{lmodern}
\ifPDFTeX\else
  % xetex/luatex font selection
\fi
% Use upquote if available, for straight quotes in verbatim environments
\IfFileExists{upquote.sty}{\usepackage{upquote}}{}
\IfFileExists{microtype.sty}{% use microtype if available
  \usepackage[]{microtype}
  \UseMicrotypeSet[protrusion]{basicmath} % disable protrusion for tt fonts
}{}
\makeatletter
\@ifundefined{KOMAClassName}{% if non-KOMA class
  \IfFileExists{parskip.sty}{%
    \usepackage{parskip}
  }{% else
    \setlength{\parindent}{0pt}
    \setlength{\parskip}{6pt plus 2pt minus 1pt}}
}{% if KOMA class
  \KOMAoptions{parskip=half}}
\makeatother
\usepackage{longtable,booktabs,array}
\usepackage{calc} % for calculating minipage widths
% Correct order of tables after \paragraph or \subparagraph
\usepackage{etoolbox}
\makeatletter
\patchcmd\longtable{\par}{\if@noskipsec\mbox{}\fi\par}{}{}
\makeatother
% Allow footnotes in longtable head/foot
\IfFileExists{footnotehyper.sty}{\usepackage{footnotehyper}}{\usepackage{footnote}}
\makesavenoteenv{longtable}
\usepackage{graphicx}
\makeatletter
\newsavebox\pandoc@box
\newcommand*\pandocbounded[1]{% scales image to fit in text height/width
  \sbox\pandoc@box{#1}%
  \Gscale@div\@tempa{\textheight}{\dimexpr\ht\pandoc@box+\dp\pandoc@box\relax}%
  \Gscale@div\@tempb{\linewidth}{\wd\pandoc@box}%
  \ifdim\@tempb\p@<\@tempa\p@\let\@tempa\@tempb\fi% select the smaller of both
  \ifdim\@tempa\p@<\p@\scalebox{\@tempa}{\usebox\pandoc@box}%
  \else\usebox{\pandoc@box}%
  \fi%
}
% Set default figure placement to htbp
\def\fps@figure{htbp}
\makeatother
\setlength{\emergencystretch}{3em} % prevent overfull lines
\providecommand{\tightlist}{%
  \setlength{\itemsep}{0pt}\setlength{\parskip}{0pt}}
\usepackage[]{natbib}
\bibliographystyle{plainnat}
\usepackage{booktabs}
\usepackage{bookmark}
\IfFileExists{xurl.sty}{\usepackage{xurl}}{} % add URL line breaks if available
\urlstyle{same}
\hypersetup{
  pdftitle={Algbric Topogy},
  pdfauthor={Ashan De Silva},
  hidelinks,
  pdfcreator={LaTeX via pandoc}}

\title{Algbric Topogy}
\author{Ashan De Silva}
\date{2026-01-12}

\usepackage{amsthm}
\newtheorem{theorem}{Theorem}[chapter]
\newtheorem{lemma}{Lemma}[chapter]
\newtheorem{corollary}{Corollary}[chapter]
\newtheorem{proposition}{Proposition}[chapter]
\newtheorem{conjecture}{Conjecture}[chapter]
\theoremstyle{definition}
\newtheorem{definition}{Definition}[chapter]
\theoremstyle{definition}
\newtheorem{example}{Example}[chapter]
\theoremstyle{definition}
\newtheorem{exercise}{Exercise}[chapter]
\theoremstyle{definition}
\newtheorem{hypothesis}{Hypothesis}[chapter]
\theoremstyle{remark}
\newtheorem*{remark}{Remark}
\newtheorem*{solution}{Solution}
\begin{document}
\maketitle

{
\setcounter{tocdepth}{1}
\tableofcontents
}
\chapter{Introduction}\label{introduction}

\subsection{What is Algberic topolgy?}\label{what-is-algberic-topolgy}

\pandocbounded{\includegraphics[keepaspectratio]{fig/fig 1.png}}
Roughly it's about tools/methods connecting \textbf{algebra} and \textbf{topology}.

\pandocbounded{\includegraphics[keepaspectratio]{fig/fig 2.png}}

The ``\emph{bridge}'' is \textbf{algebraic topology}.

In order to be useful, the associations \[ X  ⇝  G(X) \quad \text{and} \quad f  ⇝  \varphi_f \] ought to satisfy some properties.

\begin{longtable}[]{@{}
  >{\raggedright\arraybackslash}p{(\linewidth - 2\tabcolsep) * \real{0.0909}}
  >{\raggedright\arraybackslash}p{(\linewidth - 2\tabcolsep) * \real{0.9091}}@{}}
\toprule\noalign{}
\begin{minipage}[b]{\linewidth}\raggedright
Property
\end{minipage} & \begin{minipage}[b]{\linewidth}\raggedright
Mathematical Form
\end{minipage} \\
\midrule\noalign{}
\endhead
\bottomrule\noalign{}
\endlastfoot
\begin{minipage}[t]{\linewidth}\raggedright
\textbf{Identity}\\
\textbf{Preservation}\strut
\end{minipage} & \(\mathrm{id}_X : X \to X \; ⇝ \; \mathrm{id}_{G(X)} : G(X) \to G(X)\) \\
\begin{minipage}[t]{\linewidth}\raggedright
\textbf{Composition}\\
\textbf{Compatibility}\strut
\end{minipage} & If \(f : X \to Y\) and \(g : Y \to Z\), then \((g \circ f) : X \to Z \; ⇝ \; \varphi_{g \circ f}=\varphi_g \circ \varphi_f : G(X) \to G(Z)\) \pandocbounded{\includegraphics[keepaspectratio]{fig/fig 3.png}} \\
\end{longtable}

In somewhat ``\emph{fancy}'' language, we are asking that the correspondence above be \textbf{functor}
So, what could we do with all this?

\begin{example}[Classical Problem]
\protect\hypertarget{exm:unnamed-chunk-1}{}\label{exm:unnamed-chunk-1}Let \(S^1 \subset D^2\) be the unit circle inside the unit disk in \(\mathbb{R}^2\).Suppose we are given a continuous map \(f : S^1 \to S^1.\) \textbf{Does there exist a continuous extension \(g : D^2 \to S^1\) such that \(g(z) = f(z)\) for all \(z \in S^1\)?}
\end{example}

\[
\begin{aligned}
& S^1 \xrightarrow{f} S^1 \\
& \downarrow i \quad \nearrow \mathbf{?~\exists} ~g \\
& D^2
\end{aligned}
\]

Imagine we had our algebraic topology tools on hand.
If there exists a continuous map \(g : D^2 \rightarrow S^1\) such that \(g \circ j = f,\) then the induced homomorphism satisfies\\
\[ \varphi_f = \varphi_{g \circ j} = \varphi_g \circ \varphi_j. \]
That is, the existence of \(g\) would imply the existence of a homomorphism \(\varphi_g\) such that the following diagram commutes:

\begin{proof}
\[
\begin{aligned}
& G(S^1) \xrightarrow{\varphi_f} G(S^1) \\
& \downarrow \varphi_j \quad \nearrow \varphi_g \\
& G(D^2)
\end{aligned}
\]

``Commutes'' mean \(\varphi_f = \varphi_g \circ \varphi_j.\)

Suppose we know,

\begin{itemize}
\tightlist
\item
  \(G(S^1) \cong \mathbb{Z}\)
\item
  \(G(D^2) \cong \{ e \}\) (the trivial group)
\item
  \(\varphi_f\) is a non-trivial homomorphism, i.e., \(\varphi_f(a) \neq 0 \text{ for all } a \in G(S^1)\) is a generator.
\end{itemize}

This yields a contradiction, since any composition through the trivial group must be trivial. Therefore, \textbf{no such map \(g\) can exist}.
\end{proof}

\textbf{Definition}: Invaraince (\emph{This is not a formal definition})
Suppose \(G\) is a correspondence of the kind we're looking for:
\[\begin{cases}
X  &⇝  G(X) \\
X \xrightarrow{f} Y  &⇝  G(X) \xrightarrow{G(f)} G(Y)
\end{cases}\]
such that:\[G(\mathrm{id}_X) = \mathrm{id}_{G(X)}, \quad G(f \circ g) = G(f) \circ G(g).\]

Then observe that it's \textbf{automatic} that

\textbf{Result} :If \(f: X \to Y\) is a homeomorphism, then\\
\(G(f): G(X) \to G(Y)\) is an isomorphism.

\begin{proof}
\leavevmode

\begin{itemize}
\tightlist
\item
  \emph{Homeomorphism}: A function \({\displaystyle f:X\to Y}\) is a homeomorphism if

  \begin{itemize}
  \tightlist
  \item
    \({\displaystyle f}\) is a bijection (one-to-one and onto),
  \item
    \({\displaystyle f}\) is continuous,
  \item
    \({\displaystyle f^{-1}}\) is continuous
  \end{itemize}
\item
  Isomorphism

  \begin{itemize}
  \tightlist
  \item
    Hormphism
  \item
    Bijective
  \end{itemize}
\end{itemize}

Let \(f^{-1}: Y \to X\) be the inverse of \(f\). Then \(f^{-1} \circ f = \mathrm{id}_X\). So,

\begin{align*}
\mathrm{id}_{G(X)} 
  &= G(\mathrm{id}_X) 
   = G(f^{-1} \circ f) 
   = G(f^{-1}) \circ G(f) \\
\mathrm{id}_{G(Y)} 
  &= G(\mathrm{id}_Y) 
   = G(f \circ f^{-1}) 
   = G(f) \circ G(f^{-1})
\end{align*}

Hence,\(G(f^{-1}) = G(f)^{-1}\).
Therefore, homeomorphic spaces \(X, Y\) yield isomorphic groups,\(G(X) \cong G(Y)\)

\emph{(We say \(G(X)\) is an \textbf{invariant} of \(X\).)}

\end{proof}

\[
\text{A necessary condition for two spaces } X, Y \text{ to be homeomorphic is that } G(X) \cong G(Y).
\]

\begin{remark}
We'll see examples showing this is \textbf{NOT} sufficient.
\end{remark}

\section{Some Basic defnitions}\label{some-basic-defnitions}

\begin{definition}[Homotopy of Maps]
\protect\hypertarget{def:unnamed-chunk-5}{}\label{def:unnamed-chunk-5}\leavevmode

Let \(X, Y\) be topological spaces, and let \(f, g : X \to Y\) be continuous maps.\\
We say that \textbf{\(f\) is homotopic to \(g\)}, and write \(f \sim g\), if there exists a continuous map\\
\[
H : X \times I \to Y
\]
such that
\[
H(x,0) = f(x), \qquad H(x,1) = g(x) \quad \text{for all } x \in X.
\]

Intuitively, the map \(H\) defines a continuous deformation of \(f\) into \(g\) over time \(t \in [0,1]\).

\textbf{Notations}:

\begin{itemize}
\tightlist
\item
  We often write \(H_t : X \to Y\) for the slice \(H(\cdot,t)\), so that\\
  \(H_0 = f, \quad H_1 = g.\)
\item
  Sometimes we write \(f \overset{H}{\sim} g\) to indicate the specific homotopy \(H\).
\end{itemize}

\end{definition}

\begin{example}
\protect\hypertarget{exm:unnamed-chunk-6}{}\label{exm:unnamed-chunk-6}\leavevmode

Let \(f, g : X \to \mathbb{R}^n\) be any continuous functions.\\
Then \(f \sim g\), because we can define a homotopy:

\[
H(x,t) = t\, g(x) + (1 - t)\, f(x)
\]

This map \(H : X \times I \to \mathbb{R}^n\) is continuous\\
(since vector space operations in \(\mathbb{R}^n\) are continuous),\\
and satisfies:

\[
H_0 = f, \qquad H_1 = g
\]

\end{example}

\begin{definition}
\protect\hypertarget{def:unnamed-chunk-7}{}\label{def:unnamed-chunk-7}Let \(A \subset X\), and suppose \(f|_A = g|_A\), i.e., \(f(a) = g(a)\) for all \(a \in A\).\\
We say that \textbf{\(f\) is homotopic to \(g\) relative to \(A\)} if there exists a homotopy

\[
H : X \times I \to Y
\]

such that

\[
H(a,t) = f(a) = g(a) \quad \text{for all } a \in A, \text{ and all } t \in [0,1].
\]
\end{definition}

\pandocbounded{\includegraphics[keepaspectratio]{fig/fig 5.png}}

\begin{example}
\protect\hypertarget{exm:unnamed-chunk-8}{}\label{exm:unnamed-chunk-8}\leavevmode

Let \(f, g : I \to \mathbb{R}^2\) be continuous maps (paths), as illustrated.

Then the interpolation homotopy is a homotopy from \(f\) to \(g\), and it satisfies

\[
H(0,t) = f(0) = g(0), \qquad
H(1,t) = f(1) = g(1)
\]

for all \(t \in [0,1]\).

Hence, \(H\) is a \textbf{homotopy relative to the endpoints}\\
\(\{0,1\} \subset I\).

\end{example}

\textbf{Notation}:

When \(f\) is homotopic to \(g\) \textbf{relative to a subset} \(A \subset X\),\\
we write:\(f \sim g \text{ rel } A\)

This means there exists a homotopy \(H : X \times I \to Y\) such that

\[
H(a,t) = f(a) = g(a) \quad \text{for all } a \in A, \, t \in [0,1].
\]

\begin{example}
\protect\hypertarget{exm:unnamed-chunk-9}{}\label{exm:unnamed-chunk-9}Let \(f, g, h : X \to Y\) be continuous maps between topological spaces.\\
We show that the relation \(f \sim g\) (homotopy) satisfies reflexivity, symmetry, and transitivity.

\begin{enumerate}
\def\labelenumi{\arabic{enumi}.}
\item
  Reflexivity
  \[
  f \sim f \quad \text{via } H(x,t) = f(x) \text{ for all } t \in [0,1].
  \]
  This is a constant homotopy.
\item
  Symmetry
  Suppose \(f \sim g\) via homotopy \(H(x,t)\).\\
  Then define:
\end{enumerate}

\[
G(x,t) = H(x,1 - t)
\]

This reverses the deformation and gives a homotopy from \(g\) to \(f\), so \(g \sim f\).

\begin{enumerate}
\def\labelenumi{\arabic{enumi}.}
\setcounter{enumi}{2}
\tightlist
\item
  Transitivity
\end{enumerate}

Suppose \(f \sim g\) via \(F(x,t)\), and \(g \sim h\) via \(G(x,t)\).\\
Define a new homotopy \(H : X \times I \to Y\) by:

\[
H(x,t) =
\begin{cases}
F(x,2t) & \text{if } 0 \le t \le \tfrac12, \\
G(x,2t - 1) & \text{if } \tfrac12 \le t \le 1.
\end{cases}
\]

Then:

\[
H(x,0) = F(x,0) = f(x), \qquad
H(x,1) = G(x,1) = h(x).
\]

By the pasting lemma (Lemma \ref{lem:PasteLemma}),\\
\(F(x,1) = g(x) = G(x,0)\), so \(H\) is continuous.\\
Thus, \(f \sim h\).
\end{example}

Let me recall pasting lemma, if you can not remember,

\begin{lemma}[Pasting Lemma]
\protect\hypertarget{lem:PasteLemma}{}\label{lem:PasteLemma}Let \(X, Y\) be both closed (or both open) subsets of a topological space \(A\) such that\\
\[
A = X \cup Y,
\]
and let \(B\) be a topological space. If \(f : A \to B\) is continuous when restricted to both \(X\) and \(Y\), then \(f\) is continuous.
\end{lemma}

\chapter{hj}\label{hj}

\bibliography{book.bib,packages.bib}

\end{document}
